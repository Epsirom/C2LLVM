\documentclass[a4paper]{article}
%\documentclass[twocolumn]{article}

\usepackage{graphicx}
\usepackage{listings}
\usepackage{xcolor}
%\usepackage{enumitem}
\usepackage{enumerate}
\usepackage{CJKutf8} %注意这里用的是CJKutf8而不是CJK
\usepackage{indentfirst}

\usepackage{tikz} % 画流程图用的
\usepackage{qtree}
\usepackage{indentfirst}%英文首行缩进
\usepackage{fancyhdr} % 排版格式
\usepackage{hyphenat} % 单词断字
\usepackage{amsmath} % for {aligned}, 公式换行
\usepackage{multicol}% 多栏排版
\usepackage{balance}% 双栏最后一页对齐
\usepackage{subfigure}% 多图
\usepackage{booktabs}% 表格画线,\toprule, \midrule, \bottomrule
\usepackage{ulem}
%\usepackage{clrscode}

% in texlive-science
\usepackage{algorithm}
\usepackage{algpseudocode}% an improvement from algorithmicx for algorithmic

\usepackage{setspace}
 
\usepackage[utf8]{inputenc}


\usepackage{xspace}

%======= XXX 要编译两遍才能有标签和引用等效果 =====%
 
\usepackage[top=2.54cm,bottom=2.54cm,left=3.17cm,right=3.17cm]{geometry} % a4paper standard
\usepackage[unicode=true]{hyperref} %注意这里不能加CJKbookmarks=true,否则会乱码
\usetikzlibrary{arrows,decorations.pathmorphing,backgrounds,positioning,fit,automata,trees}

\hypersetup{
    pdfauthor={ouoline},
    %pdftitle={test},
    %pdfsubject={Subject},
    %pdfkeywords={Keyword1, Keyword2, ...},
    %pdfcreator={LaTeX with hyperref package},
    %pdfproducer = {dvips + ps2pdf},
    %bookmarksnumbered=true,
    %colorlinks=no,
    pdfborder={0 0 0},
    %bookmarksopen=true,
}
%------------------------------------------------------------------%
 
\setcounter{secnumdepth}{5} % 编号的深度,4 表示到 paragraph 一级
%\setcounter{tocdepth}{4} % 目录中的深度
 
%------------------------------------------------------------------%
\usepackage{color}
\definecolor{lightgray}{rgb}{.9,.9,.9}
\definecolor{darkgray}{rgb}{.4,.4,.4}
\definecolor{purple}{rgb}{0.65, 0.12, 0.82}

\lstdefinelanguage{JavaScript}{
  keywords={typeof, new, true, false, catch, function, return, null, catch, switch, var, if, in, while, do, else, case, break},
  keywordstyle=\color{blue}\bfseries,
  ndkeywords={class, export, boolean, throw, implements, import, this},
  ndkeywordstyle=\color{darkgray}\bfseries,
  identifierstyle=\color{black},
  sensitive=false,
  comment=[l]{//},
  morecomment=[s]{/*}{*/},
  commentstyle=\color{purple}\ttfamily,
  stringstyle=\color{red}\ttfamily,
  morestring=[b]',
  morestring=[b]"
}

\lstset{% general command to set parameter(s)
        language={JavaScript},
        %numbers=left,
        basicstyle=\tt, % 默认对所有字符使用等宽字体
        keywordstyle=\color{blue},%\bfseries\underbar% underlined bold black keywords
        identifierstyle=,           % nothing happens
        stringstyle=\color{purple},
        commentstyle=\color{gray},
        escapechar=`,
        showstringspaces=false,     % no special string spaces
        breaklines=true, % 自动换行
    }


%------------------------------------------------------------------%
 
% 自己定义新命令,参数依次是
% \newcommand{新命令名称(带反斜线)}[参数个数(最多9个)]{命令定义}
% 实际上相当于宏替换
% \newcommand{\sayhelloto}[1]{hello,#1}
\newcommand{\template}[3] {
    \item \textbf{#1}
    \begin{enumerate}
        \item[\textbf{#2}] #3 
    \end{enumerate}
}
\newcommand{\jie}[2]{\template{#1}{解}{#2}}
\newcommand{\zheng}[2]{\template{#1}{证明}{#2}}
 
%------------------------------------------------------------------%
 
% 放在导言区,设置全局行距
\linespread{1.6}
 
% 放在导言区,公式编号和章节相关
%\makeatletter % `@' now normal ``letter''
%\@addtoreset{equation}{section}
%\makeatother % `@' is restored as ``non-letter''
%\renewcommand\theequation{\oldstylenums{\thesection}%
%.\oldstylenums{\arabic{equation}}}


\begin{document}

\begin{CJK*}{UTF8}{gbsn}
    \CJKindent
    \setlength{\parindent}{2em} % no indent
 
    \pagestyle{fancy}
  
    %\begin{center}
    %\Huge{title}
    %\vspace{25pt} % 25pt between title and text
    %\end{center}
    \title{\huge{计算机与网络体系结构(2)}\\\Large{编译原理}\\{\large Project 1:使用自动机分析Bib\TeX\xspace文件}}
    \author{软件11\hspace{10pt}陈华榕\hspace{10pt}2011013236}
    \date{\today}
    \maketitle
    \tableofcontents    
    \newpage

    \section{实验背景}
    \subsection{实验环境}
    如需复现实验结果请准备一台PC机并安装Windows/OS X/Linux等桌面系统,然后安装$>=0.10.24$版本的\textit{Node}。

    \subsection{目录结构}
{
\tikzstyle{every node}=[draw=black,thick,anchor=west]
\tikzstyle{selected}=[draw=red,fill=red!30]
\tikzstyle{optional}=[dashed,fill=gray!50]
\begin{tikzpicture}[%
  grow via three points={one child at (0.5,-0.7) and
  two children at (0.5,-0.7) and (0.5,-1.4)},
  edge from parent path={(\tikzparentnode.south) |- (\tikzchildnode.west)}]
  \node {2011013236}
    child { node [optional] {source $|$ 源码}
        child { node [optional] {BibJS}
            child { node [optional] {lib $|$ 库文件}
                child { node [optional] {automata $|$ 自动机\footnotemark[1]}
                    child { node {inner-data.js $|$ 定义自动机的数据}}
                    child { node {runner.js $|$ 自动机的运行器}}
                }
                child [missing] {}
                child [missing] {}
                child { node [optional] {scanner $|$ 词法单元扫描器}
                    child { node {token-types.js $|$ 支持的词法单元类型}}
                    child { node {char-token.js $|$ 字符单元}}
                    child { node {token.js $|$ 词法单元}}
                    child { node {scanner.js $|$ 扫描器}}
                }
                child [missing] {}
                child [missing] {}
                child [missing] {}
                child [missing] {}
                child { node {bibdad.js $|$ Bib解析器}}
            }
                child [missing] {}
                child [missing] {}
                child [missing] {}
                child [missing] {}
                child [missing] {}
                child [missing] {}
                child [missing] {}
                child [missing] {}
                child [missing] {}
            child { node [optional] {tests $|$ 自动化测试}
                child { node [optional] {data $|$ 测试数据}
                    child { node {example.bib $|$ 测试文件}}
                }
                child [missing] {}
                child { node [selected] {test-bib.js $|$ 测试执行入口,可通过npm调用}}
            }
                child [missing] {}
                child [missing] {}
                child [missing] {}
            child { node {index.js $|$ 库入口}}
    child { node {... $|$ 其他文件,包括定义库的package.json等}}
        }
    }
                child [missing] {}
                child [missing] {}
                child [missing] {}
                child [missing] {}
                child [missing] {}
                child [missing] {}
                child [missing] {}
                child [missing] {}
                child [missing] {}
                child [missing] {}
                child [missing] {}
                child [missing] {}
                child [missing] {}
                child [missing] {}
                child [missing] {}
                child [missing] {}
                child [missing] {}
    child { node {report.pdf $|$ 本报告}}
    child { node {README.txt $|$ 请先阅读此文件}}
    child { node {result.txt $|$ 本实验所给example.bib的解析结果}}
;
\end{tikzpicture}
}
\footnotetext[1]{无$\epsilon$转移的单状态非确定型有限自动机}

    \subsection{完成情况与测试}
    \label{sec:abouttest}
    本实验中,通过Node.js完成了一个Bib文件解析库BibJS,可作为包供第三方Node.js应用引入。还借助这个库完成了符合实验要求的测试程序。由于所用编程语言是脚本语言,因此不需要提交可执行的版本。
    \par 本次作业提交在了Github上,在作业提交的截止时间前以私有库存在,作业提交截止时间后就修改为公开库,git版本库的地址为:\href{https://github.com/Epsirom/BibJS}{https://github.com/Epsirom/BibJS}。
    \par 如果在安装了npm的机器上进行测试,可以直接在BibJS目录下运行\lstinline[language=sh]{npm test},将以./tests/data/example.bib为输入,处理结果将输出到./tests/data/result.txt。
    \par 如果在未安装npm的机器上进行测试,请以BibJS目录为工作目录运行./tests/test-bib.js,即在BibJS目录下运行\lstinline{node ./tests/test-bib.js}。
    \par 运行测试程序,会将带调试信息的结果输出在命令行,而输出到result.txt的则不带调试信息只保留有效的解析结果。

    \section{实验分析}
    \subsection{整体思路}
    我将本次实验目标分为两部分:第一部分通过构建自动机A由字符串分析词法单元;第二部分通过构建自动机B由词法单元构建Bib数据对象。进行Bib解析时,先通过自动机A分析出一个词法单元,将该词法单元作为自动机B的一次输入,然后再通过自动机A分析出一个词法单元作为自动机B的新一次输入,不断进行下去直到发生错误或完整解析了一个Bib对象。
    \par 两部分所用的自动机都是“无$\epsilon$转移的单状态非确定型有限自动机”,这种自动机是我自行定义的,兼顾DFA的实现简单和NFA的一部分灵活性,能比较方便地应用在本次实验中。
    \par 实际上本实验的要求可以通过先编写一个涵盖自动机和词法分析的Bib解析库,然后借助该库对输入文件进行分析并按格式输出结果,这样一种方式来完成。我正是这么做的。

    \subsection{无$\epsilon$转移的单状态非确定型有限自动机}
    由于该自动机与DFA和NFA都有一定相似性,因此这里不给出公理化的定义了。\\\sout{(其实是因为公理化定义太麻烦了)}
    \par 您可以认为该自动机就是一个DFA,只是它的每个状态都不一定给出字母表中所有可能输入的转移,如果在某状态下输入某字符的转移没有定义,将直接拒绝整个串——这样的方式和NFA是一样的。因此实际上该自动机具有NFA的特征,可称为“非确定型有限自动机”,但它和DFA一样只有单状态,同时没有$\epsilon$转移。
    \par 这样的自动机是本实验实现的核心所在,各个步骤实际上都是在构建这种自动机来进行状态转移与计算。
    \subsubsection{自动机数据结构}
    通过BibJS目录下的./lib/automata/inner-data.js进行自动机的数据结构管理。在其中定义了类AutomataData,有三个数据成员:nodes,start,status。
    \par 本自动机以字符串作为状态,每个状态的信息记录在nodes中,每个状态都是nodes的一个key,其对应的value是状态的具体设定。start记录的是开始状态。status记录的是当前状态。
    \par nodes中的状态需要保存三个值:next,accept,back。next是一个对象,accept是布尔值,back是个整数。accept记录这个状态是不是自动机的接受状态。back记录到达这个状态后要把字符串的扫描指针前移多少字节,可视为状态的附加信息。next中每个key都是一个input,其value记录当前状态接收到这样的input后需要转移到哪个新状态。

    \subsubsection{自动机执行器}
    通过BibJS目录下的./lib/automata/runner.js驱动自动机的执行。在其中定义了类AutomataRunner,其中管理AutomataData对象并提供重启(restart)、单步执行(executeOnce)等方法。

    \subsection{解析词法单元}
    \subsubsection{实现概况}
    这部分只负责根据输入串分析出词法单元的类型及其属性。所用输入字符会先封装成CharToken(BibJS/lib/scanner/char-token.js),然后将CharToken的类型作为自动机的输入去驱动自动机的单步执行。每当自动机执行到可接受状态时就说明已经解析出了一个Bib的词法单元Token(BibJS/lib/scanner/token.js),可以返回给调用者并将自动机重启以继续解析下一个词法单元(这样虽然看起来不总是有效的,但它对Bib格式总是有效的)。

    \subsubsection{数据结构}
    CharToken是单字符的封装词法单元,支持以下类型(这些类型定义在BibJS/lib/scanner/token-types.js中):\par
    \begin{tabular}
        {|c|c|c|}
        \hline
        类型 & 简写 & 说明\\\hline
        TYPE\_START & TS & @号\\\hline
        EMPTY\_CHAR & EC & 空白符,包括空格、换行、水平制表符、垂直制表符等\\\hline
        EQUAL & EQ & 等号\\\hline
        COMMA & CM & 逗号\\\hline
        QUOTE & QT & 单引号\\\hline
        DOUBLE\_QUOTE & DQ & 双引号\\\hline
        LEFT\_BRACE & LB & 左大括号\\\hline
        RIGHT\_BRACE & RB & 右大括号\\\hline
        COMMON\_CHAR & CC & 其他字符\\\hline
        TOO\_LONG & TL & 非单字符,太长了\\\hline
        TOO\_SHORT & TS & 非单字符,太短了\\\hline
        ERROR & ER & 出错了\\\hline
    \end{tabular}
    \par 这里的简写在下面描述自动机时用到,能让自动机图更美观\sout{(还能少打很多字)}。
    \par Token用于存放解析出的Bib词法单元,支持以下类型(同样定义在BibJS/lib/scanner/token-types.js中):\par 
    \begin{tabular}
        {|c|c|}
        \hline
        类型 & 说明\\\hline
        CONFERENCE等 & 解析出了这个Bib对象的类型\\\hline
        LEFT\_BRACE & 左大括号\\\hline
        CITE\_NAME & 这个Bib对象的引用名,在输出中不显示\\\hline
        AUTHOR等 & 解析出的Bib对象的具体属性\\\hline
        RIGHT\_BRACE & 右大括号\\\hline
        OTHER & 解析出的未知Bib对象的属性\\\hline
        NO\_MORE & 已经完整扫完了输出串\\\hline
        UNSUPPORT\_LABEL & 解析出的未知Bib对象的属性\\\hline
        UNSUPPORT\_TYPE & 解析出的未知Bib对象类型,类似CONFERENCE这样的\\\hline
        ERROR & 出错了\\\hline
    \end{tabular}
    \par Scanner是解析词法单元的主类,定义在BibJS/lib/scanner/scanner.js中,其中有token, str, pointer, automata等数据成员。token是供调用者获取的解析结果,str是输入字符串,pointer是输入串的指针(指示当前扫描到输入串的哪个字符),automata是所用的自动机。
    \par Scanner中的scan方法是核心所在,在其中先让自动机重启,然后在未出错且未接受的情况下将输入串pointer处的字符封装成CharToken作为输入提供给自动机去单步执行。若执行出错,就设token为ERROR。解析过程中会根据当前的状态对token进行一定设置,因此若转移到了接受状态,就能直接返回了。在自动机单步执行过程中,会对当前状态的back属性有一定响应,具体就是每执行一步就先将pointer自增1,然后根据back调整一次pointer。

    \subsubsection{自动机}
    如上所述,输入字符集为所有可能的CharToken类型。
    \par 自动机如图\ref{fig:automata1}。
    \par 接着讨论一下自动机中各个可接受状态可能会解析得到的Bib词法单元,如下表所示:\\
    \begin{tabular}
        {|c|c|c|}
        \hline
        可接受状态 & 可能解析得到的Bib词法单元 & 说明\\\hline
        type\_end & CONFERENCE,PROCEEDINGS等 & 解析出了以@打头的Bib对象类型\\\hline
        left\_brace & LEFT\_BRACE & 左大括号\\\hline
        right\_brace & RIGHT\_BRACE & 右大括号\\\hline
        label\_end & AUTHOR,TITLE等Bib对象属性 & 通过Bib属性的定义方式得到\\\hline
        label\_end\_no\_back & 同label\_end & 只是back=0罢了\\\hline
    \end{tabular}
    \begin{figure}
    \begin{tikzpicture}[>=stealth',shorten >=1pt,auto,node distance=4cm]
      \node[initial,state]  (init)                          {init};
      \node[state]          (ts)    [above right of=init]   {type\_start};
      \node[state,accepting]          (te)    [right of=ts]           {*type\_end};
      \node[state,accepting](lb)    [right of=init]         {left\_brace};
      \node[state,accepting](rb)    [below right of=init]   {right\_brace};
      \node[state]          (ls)    [below of=init]         {label\_start};
      \node[state,accepting](le)    [below of=ls]      {*label\_end};
      \node[state,accepting](lenb)  [below left of=ls]           {label\_end\_no\_back};
      \node[state]          (lr)    [below right of=ls]     {label\_ready};
      \node[state]          (lvr)   [right of=lr]           {label\_value\_ready};
      \node[state]          (dv)    [below left of=lvr]     {dq\_value};
      \node[state]          (bv)    [below right of=lvr]    {brace\_value};
      \node[state]          (lg)    [below of=lvr]          {label\_good};


      \path[->] (init)      edge    [loop above]    node    {EC}    (init)
                            edge                    node    {TS}    (ts)
                            edge                    node    {LB}    (lb)
                            edge                    node    {RB}    (rb)
                            edge                    node    {CC}    (ls)
                (ts)        edge    [loop above]    node    {CC}    (ts)
                            edge                    node    {EC $|$ LB} (te)
                (ls)        edge    [loop left]     node    {CC}    (ls)
                            edge                    node    {CM}    (lenb)
                            edge                    node    {RB}    (le)
                            edge                    node    {EQ}    (lvr)
                            edge                    node    {EC}    (lr)
                (lr)        edge    [loop below]    node    {EC}    (lr)
                            edge                    node    {EQ}    (lvr)
                            edge    [bend right=10]                node    {CM}    (lenb)
                            edge                    node    {LB}    (le)
                (lvr)       edge    [loop above]    node    {EC}    (lvr)
                            edge                    node    {DQ}    (dv)
                            edge                    node    {LB}    (bv)
                (dv)        edge    [loop left]     node    {其他字符} (dv)
                            edge                    node    {DQ}    (lg)
                (bv)        edge    [loop above]    node    {其他字符} (bv)
                            edge                    node    {RB}    (lg)
                (lg)        edge    [bend left=60]                node    {CM}    (lenb)
                            edge    [bend left=50]                node    {RB}    (le);
    \end{tikzpicture}
        \caption{解析语法单元的自动机。其中带星号的状态表示该状态的back为1,不带星号表示back为0}
        \label{fig:automata1}
    \end{figure}

    \subsection{解析Bib}
    \subsubsection{实现概况}
    这部分借助于前面实现的Scanner对字符串进行Bib词法单元解析并保存,实现了BibDad类,在BibJS/lib/bibdad.js中。

    \subsubsection{数据结构}
    BibDad的数据成员包括scanner,bibtype,data,automata。
    \par scanner是一个Scanner对象,bibtype为解析出的Bib对象的类型(比如CONFERENCE),\\data以key-value的方式存储Bib对象的各种属性,automata是这部分用到的自动机。

    \subsubsection{自动机}
    输入字符集为所有可能的Token类型。
    \par 自动机如图\ref{fig:automata2}。
    \par 当自动机转移到finish状态,说明已经完成了一个Bib对象的解析。这之后可以重启自动机继续解析下一个Bib对象。
    \par BibDad的主要成员方法包括analyze,validata,output。在analyze中,会先重置自动机,然后通过Scanner不断读取词法单元,根据得到的词法单元驱动自动机,直至解析出错或完整解析了一个Bib对象。validate中会根据解析到的Bib对象类型(CONFERENCE等)检查该类型的必要属性是否都已具备。output中会根据解析到的Bib对象属性返回该对象的输出字符串(也就是符合实验要求的输出字符串,未来可考虑重写这部分使之支持完整的Bib格式输出)。
    \par 在输出处理AUTHOR时,还通过第三个自动机将其中的“~and~”(前后各有1个空格)替换为“,~”(末位有1个空格),其实就是简单的字符串匹配自动机,这里省略。

    \begin{figure}
    \begin{tikzpicture}[>=stealth',shorten >=1pt,auto,node distance=4cm]
      \node[initial,state]  (init)                          {init};
      \node[state]          (tr)        [right of=init]     {type\_ready};
      \node[state]          (p)         [right of=tr]       {processing};
      \node[state,accepting](f)         [right of=p]        {finish};


      \path[->] (init)      edge                    node    {CF $|$ IP $|$ PC}    (tr)
                (tr)        edge                    node    {LB}   (p)
                (p)         edge    [loop above]    node    {AUTHOR,TITLE等Bib属性}  (p)
                            edge                    node    {RB}  (f);
    \end{tikzpicture}
        \caption{解析Bib的自动机。CF表示CONFERENCE,IP表示INPROCEEDINGS,PC表示PROCEEDINGS,LB表示LEFT\_BRACE,RB表示RIGHT\_BRACE。}
        \label{fig:automata2}
    \end{figure}

    \subsection{自动化测试}
    \subsubsection{实现概况}
    完成了一个自动化测试实例BibJS/tests/test-bib.js,运行方法见\ref{sec:abouttest}。该测试中会依照实验要求将example.bib读入,然后将解析结果输出至result.txt。这两个文件都放在BibJS/tests/data目录下。

    \subsubsection{自动化测试步骤}
    在该自动化测试中,先从文件中读取输入字符串,然后用它构建BibDad对象,接着不断调用该对象analyze方法,若该方法返回true,说明解析成功,就进行validate,如果仍通过validate就调用output将结果输出到文件。
    \par 若analyze返回false,说明解析失败,判断此时是完成了解析还是解析过程中出错,如果是解析过程中出错,就打印错误信息(会打印出在输入串的第几个字节解析出错,并打印该字节附近以15字节为半径的字符串,空白符会全部替换为空格),错误信息的显示效果还算不错。

    \section{实验结果}
    本实验提供的example.bib的解析结果见result.txt。
    \par 在控制台的输出与result.txt不同,还带有调试信息,您可以试试按\ref{sec:abouttest}的步骤运行。
    \par 也可以试试修改BibJS/tests/data/example.bib使之不符合Bib规范,再运行看看控制台是怎样输出错误信息的。

\clearpage
\end{CJK*}
\end{document}


