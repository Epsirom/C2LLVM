\documentclass[a4paper]{article}
%\documentclass[twocolumn]{article}

\usepackage{graphicx}
\usepackage{listings}
\usepackage{xcolor}
%\usepackage{enumitem}
\usepackage{enumerate}
\usepackage{CJKutf8} %注意这里用的是CJKutf8而不是CJK
\usepackage{indentfirst}

\usepackage{tikz} % 画流程图用的
\usepackage{qtree}
\usepackage{indentfirst}%英文首行缩进
\usepackage{fancyhdr} % 排版格式
\usepackage{hyphenat} % 单词断字
\usepackage{amsmath} % for {aligned}, 公式换行
\usepackage{multicol}% 多栏排版
\usepackage{balance}% 双栏最后一页对齐
\usepackage{subfigure}% 多图
\usepackage{booktabs}% 表格画线,\toprule, \midrule, \bottomrule
\usepackage{ulem}
%\usepackage{clrscode}

% in texlive-science
\usepackage{algorithm}
\usepackage{algpseudocode}% an improvement from algorithmicx for algorithmic

\usepackage{setspace}
 
\usepackage[utf8]{inputenc}


\usepackage{xspace}

%======= XXX 要编译两遍才能有标签和引用等效果 =====%
 
\usepackage[top=2.54cm,bottom=2.54cm,left=3.17cm,right=3.17cm]{geometry} % a4paper standard
\usepackage[unicode=true]{hyperref} %注意这里不能加CJKbookmarks=true,否则会乱码
\usetikzlibrary{arrows,decorations.pathmorphing,backgrounds,positioning,fit,automata,trees}

\hypersetup{
    pdfauthor={ouoline},
    %pdftitle={test},
    %pdfsubject={Subject},
    %pdfkeywords={Keyword1, Keyword2, ...},
    %pdfcreator={LaTeX with hyperref package},
    %pdfproducer = {dvips + ps2pdf},
    %bookmarksnumbered=true,
    %colorlinks=no,
    pdfborder={0 0 0},
    %bookmarksopen=true,
}
%------------------------------------------------------------------%
 
\setcounter{secnumdepth}{5} % 编号的深度,4 表示到 paragraph 一级
%\setcounter{tocdepth}{4} % 目录中的深度
 
%------------------------------------------------------------------%
\usepackage{color}
\definecolor{lightgray}{rgb}{.9,.9,.9}
\definecolor{darkgray}{rgb}{.4,.4,.4}
\definecolor{purple}{rgb}{0.65, 0.12, 0.82}

\lstdefinelanguage{JavaScript}{
  keywords={typeof, new, true, false, catch, function, return, null, catch, switch, var, if, in, while, do, else, case, break},
  keywordstyle=\color{blue}\bfseries,
  ndkeywords={class, export, boolean, throw, implements, import, this},
  ndkeywordstyle=\color{darkgray}\bfseries,
  identifierstyle=\color{black},
  sensitive=false,
  comment=[l]{//},
  morecomment=[s]{/*}{*/},
  commentstyle=\color{purple}\ttfamily,
  stringstyle=\color{red}\ttfamily,
  morestring=[b]',
  morestring=[b]"
}

\lstset{% general command to set parameter(s)
        language={JavaScript},
        %numbers=left,
        basicstyle=\tt, % 默认对所有字符使用等宽字体
        keywordstyle=\color{blue},%\bfseries\underbar% underlined bold black keywords
        identifierstyle=,           % nothing happens
        stringstyle=\color{purple},
        commentstyle=\color{gray},
        escapechar=`,
        showstringspaces=false,     % no special string spaces
        breaklines=true, % 自动换行
    }


%------------------------------------------------------------------%
 
% 自己定义新命令,参数依次是
% \newcommand{新命令名称(带反斜线)}[参数个数(最多9个)]{命令定义}
% 实际上相当于宏替换
% \newcommand{\sayhelloto}[1]{hello,#1}
\newcommand{\template}[3] {
    \item \textbf{#1}
    \begin{enumerate}
        \item[\textbf{#2}] #3 
    \end{enumerate}
}
\newcommand{\jie}[2]{\template{#1}{解}{#2}}
\newcommand{\zheng}[2]{\template{#1}{证明}{#2}}
 
%------------------------------------------------------------------%
 
% 放在导言区,设置全局行距
\linespread{1.6}
 
% 放在导言区,公式编号和章节相关
%\makeatletter % `@' now normal ``letter''
%\@addtoreset{equation}{section}
%\makeatother % `@' is restored as ``non-letter''
%\renewcommand\theequation{\oldstylenums{\thesection}%
%.\oldstylenums{\arabic{equation}}}


\begin{document}

\begin{CJK*}{UTF8}{gbsn}
    \CJKindent
    \setlength{\parindent}{2em} % no indent
 
    \pagestyle{fancy}
  
    %\begin{center}
    %\Huge{title}
    %\vspace{25pt} % 25pt between title and text
    %\end{center}
    \title{\huge{计算机与网络体系结构(2)}\\\Large{编译原理}\\{\large 简单编译器的实现:将C语言编译为LLVM}}
    \author{
    软件11\hspace{10pt}陈 璐\hspace{10pt}2011013249\\
    软件12\hspace{10pt}王肖佑\hspace{10pt}2011013273\\
    软件11\hspace{10pt}陈华榕\hspace{10pt}2011013236
    }
    \date{\today}
    \maketitle
    \tableofcontents    
    \newpage

    \section{实验背景}
    \subsection{实验环境}
    如需复现实验结果请准备一台PC机并安装Windows/OS X/Linux等桌面系统,然后安装$>=0.10.24$版本的\textit{Node}。
    \par 如需将实验结果编译为可执行程序,需要一台PC机并安装Linux桌面系统,同时通过\textit{apt-get}安装最新版本的\textit{clang}。

    \subsection{目录结构}
{
\tikzstyle{every node}=[draw=black,thick,anchor=west]
\tikzstyle{selected}=[draw=red,fill=red!30]
\tikzstyle{optional}=[dashed,fill=gray!50]
\begin{tikzpicture}[%
  grow via three points={one child at (0.5,-0.7) and
  two children at (0.5,-0.7) and (0.5,-1.4)},
  edge from parent path={(\tikzparentnode.south) |- (\tikzchildnode.west)}]
  \node {C2LLVM}
    child { node [optional] {source $|$ 源码}
        child { node [optional] {C2LLVM}
            child { node [optional] {lib $|$ 库文件}
                child { node [optional] {parser $|$ 词法语法分析器}
                    child { node {parser-entry.js $|$ 分析器类CParser}}
                    child { node {SimpleCv4.g $|$ 语法文件}}
                    child { node {SimpleCv4.tokens $|$ 语法的Token文件}}
                    child { node {SimpleCv4Lexer.js $|$ 词法分析}}
                    child { node {SimpleCv4Parser.js $|$ 语法分析}}
                }
                child [missing] {}
                child [missing] {}
                child [missing] {}
                child [missing] {}
                child [missing] {}
                child { node [optional] {inspector $|$ 语法语义检查器}
                    child { node {id-inspector.js $|$ 变量作用域检查}}
                }
                child [missing] {}
                child { node [optional] {generator $|$ 目标代码生成器}
                    child { node {LLVM.js $|$ 生成LLVM代码}}
                }
                child [missing] {}
                child { node {antlr-3.3.jar $|$ ANTLR分析工具}}
                child { node {errmgr.js $|$ 错误管理器,保存错误信息,格式化输出}}
                child { node {compiler.js $|$ Compiler类}}
            }
            child [missing] {}
            child [missing] {}
            child [missing] {}
            child [missing] {}
            child [missing] {}
            child [missing] {}
            child [missing] {}
            child [missing] {}
            child [missing] {}
            child [missing] {}
            child [missing] {}
            child [missing] {}
            child [missing] {}
            child { node [optional] {tests $|$ 自动化测试}
                child { node [optional] {data $|$ 测试数据}
                    child { node {arraylist.c $|$ 测试文件}}
                    child { node {arraylist.ll $|$ 使用clang编译为LLVM的结果}}
                }
                child [missing] {}
                child [missing] {}
                child { node [selected] {test-main.js $|$ 测试执行入口,可通过npm调用}}
            }
            child [missing] {}
            child [missing] {}
            child [missing] {}
            child [missing] {}
            child { node {index.js $|$ 库入口}}
            child { node {... $|$ 其他文件,包括定义库的package.json等}}
        }
    }
    child [missing] {}
    child [missing] {}
    child [missing] {}
    child [missing] {}
    child [missing] {}
    child [missing] {}
    child [missing] {}
    child [missing] {}
    child [missing] {}
    child [missing] {}
    child [missing] {}
    child [missing] {}
    child [missing] {}
    child [missing] {}
    child [missing] {}
    child [missing] {}
    child [missing] {}
    child [missing] {}
    child [missing] {}
    child [missing] {}
    child [missing] {}
    child [missing] {}
    child { node {report.pdf $|$ 本报告}}
    child { node {result.ll $|$ 本编译器对tests/data/arraylist.c的编译结果}}
;
\end{tikzpicture}
}

    \subsection{完成情况与测试}
    \label{sec:abouttest}
    本实验中,通过Node.js完成了一个以C语言为前端语言,LLVM为后端语言的编译库C2LLVM,可作为包供第三方Node.js应用引入。还借助这个库完成了符合实验要求的测试程序。由于所用编程语言是脚本语言,因此不需要提交可执行的版本。
    \par 本次作业提交在了Github上,在作业提交的截止时间前以私有库存在,作业提交截止时间后就修改为公开库,git版本库的地址为:\href{https://github.com/Epsirom/C2LLVM}{https://github.com/Epsirom/C2LLVM}。
    \par 如果在安装了npm的机器上进行测试,可以直接在C2LLVM目录下运行\lstinline[language=sh]{npm test},将以./tests/data/arraylist.c为输入,处理结果将输出到./tests/data/result.ll(该文件的副本已经提供在提交目录的根目录中)。
    \par 如果在未安装npm的机器上进行测试,请以C2LLVM目录为工作目录运行./tests/test-main.js,即在./source/C2LLVM目录下运行\lstinline{node ./tests/test-main.js}。
    \par 运行测试程序,会将调试信息输出在命令行,而输出到result.ll的则不带调试信息只保留有效的解析结果。
    \par 在安装有clang的Linux机器上,可以通过\lstinline[language=sh]{clang -o result result.ll}来进行编译,然后通过\lstinline[language=sh]{./result}来运行。

    \section{实验分析}
    \subsection{整体思路}
    我们将实验分为三个部分:第一部分借助ANTLR和手动编写的语法文件进行词法语法分析并构建语法分析树AST;第二部分通过手动遍历AST的方式进行语法及语义检查;第三部分仍是通过手动遍历AST的方式进行目标代码生成。
    \par 实际上借助于ANTLR工具和手动的编码能完成以上三部分的全部工作,但后两部分我们仍决定通过手动遍历AST的方式来实现,以此减少最终代码量,希望能提高效率。
    \par 我们的编译器支持大量语法特性,依照要求本应针对每个特性写一个测试程序,但为简单起见,我们只提供了arraylist.c这一测试程序,它对应的是实验要求中输入程序c,也就是array-list。在arraylist.c中不仅涵盖实验要求中的类(考虑C语言的特性,用结构体来替代)、成员变量、成员函数(考虑C语言的特性,用包含结构体变量作为参数的全局函数来替代)、动态内存分配等,还包括类型定义、动态内存释放、内存段拷贝、数组、显式类型转换、隐式类型转换、递归、多种循环体等在内的大量语法特性,基本能说明我们的编译器拥有较好的编译能力。

    \subsection{词法、语法分析}

    \subsubsection{Antlr工具}

    \subsubsection{抽象语法分析树AST}

    \subsection{语法及语义检查}
    \subsubsection{实现概况}
    这部分应该遍历AST进行相应的检查,至少应包括变量、函数、类型等在内的不同标识符的作用域检查、函数参数列表检查等。考虑到每一种标识符的检查都是复杂繁琐的,由于时间有限,我们只对变量进行了作用域检查,其他的语法语义大同小异。

    \subsubsection{变量作用域检查}
    维护一张变量表paramTable,是普通的js对象,作为key-value映射表使用,以变量名作为key,而value则是该变量的作用域表。
    \par 每个变量的作用域表都是一个数组,其中填充布尔值,表示数组下标对应的作用域等级是否有该名称变量的定义,true表示有。
    \par 对于C语言的不同作用域,其作用域等级可能不同。全局作用域的作用域等级为1,每进入一个函数或一个BLOCK作用域(也就是大括号限定的作用域),作用域等级+1;每完成一个函数或BLOCK,释放该作用域的所有变量并恢复作用域等级。
    \par 当在特定作用域声明一个变量时,若该变量在当前作用域已经存在,就报错,形如:
    \begin{verbatim}
existed: ParamName
    \end{verbatim}
    \par 当在特定作用域中调用了一个变量时,若该变量在变量表中没有定义(如果某个变量释放后其对应的作用域表均为false,会删除该变量的作用域表,这样在变量表中没有定义的变量就是在当前作用域不可用的),就报错,形如:
    \begin{verbatim}
not found: ParamName
    \end{verbatim}

    \subsection{目标代码生成}
    \subsubsection{实现概况}
    前面我们已经完成了AST的构建并通过自顶向下的方式遍历AST完成了语法及语义的检查。到了目标代码生成阶段,虽然我们进行的语法及语义检查并不完善,但现在可以认为AST中不存在语法及语义错误,实际上在目标代码生成阶段就应该做这样的假设。
    \par 进行目标代码生成同样可以直接通过编写tree grammar来进行,但我们仍采用了手动遍历的方式进行,虽然增加了手动编码量,但减少了最终代码量,希望能以此提高效率。
    \par 经过努力,我们支持了在编译过程中保持大量语言特性,甚至已经拥有编译稍微复杂的C语言程序的能力。

    \subsubsection{自顶向下基于模板的目标代码生成}

    \subsubsection{目前支持的C语言特性}
    \begin{enumerate}
        \item 数据类型与变量
        \begin{enumerate}
            \item 支持结构体、数组、指针等;
            \item 但不支持指针的指针;
            \item 结构体成员也支持数组、指针等;
            \item 支持字符串常量;
            \item 支持整型常量;
            \item 支持局部变量(即非全局变量,也就是有一定作用域的变量)。
        \end{enumerate}
        
        \item 类型转换
        \begin{enumerate}
            \item 支持显式类型转换、隐式类型转换;
            \item 支持特殊的指针转整型:ptrtoint;
            \item 支持typedef。
        \end{enumerate}
        
        \item 函数
        \begin{enumerate}
            \item 支持函数定义、调用;
            \item 支持无返回值(void)的函数;
            \item 支持printf、malloc、free、memcpy等系统函数;
            \item 自动进行隐式类型转换。
        \end{enumerate}
        
        \item 运算
        \begin{enumerate}
            \item 赋值(会进行隐式类型转换);
            \item 支持加减乘除等常用算术运算;
            \item 自增(++)、自减(--);
            \item 支持常用逻辑运算。
        \end{enumerate}

        \item 选择语句
        \begin{enumerate}
            \item 支持;
            \item 还支持选择语句的嵌套(形如if...else{if...else...})。
        \end{enumerate}

        \item 循环语句:支持for和while。
    \end{enumerate}

    \subsection{自动化测试}
    \subsubsection{实现概况}
    完成了一个自动化测试实例BibJS/tests/test-bib.js,运行方法见\ref{sec:abouttest}。该测试中会依照实验要求将example.bib读入,然后将解析结果输出至result.txt。这两个文件都放在BibJS/tests/data目录下。

    \subsubsection{自动化测试步骤}
    在该自动化测试中,先从文件中读取输入字符串,然后用它构建BibDad对象,接着不断调用该对象analyze方法,若该方法返回true,说明解析成功,就进行validate,如果仍通过validate就调用output将结果输出到文件。
    \par 若analyze返回false,说明解析失败,判断此时是完成了解析还是解析过程中出错,如果是解析过程中出错,就打印错误信息(会打印出在输入串的第几个字节解析出错,并打印该字节附近以15字节为半径的字符串,空白符会全部替换为空格),错误信息的显示效果还算不错。

    \section{实验结果}
    本实验提供的arraylist.c的解析结果见result.ll。
    \par arraylist.c中定义了结构体ArrayList,实现了创建、释放、扩增容量、插入数据、删除数据(两种方式:根据索引、根据值)、排序、功能测试等函数。在其中频繁进行选择判断、循环、函数调用等,其复杂性基本能说明我们的编译器已经对较复杂C语言程序有不错的支持。但其中有些地方仍按照奇怪的方式来书写,比如为了避免类似\lstinline{char *a, *b, c;}这样语句造成的处理难度,我们的编译器只支持\lstinline{char *a; char *b; char c;}这样的方式来定义变量。
    \par 值得一提的是,clang的编译结果(tests/data/arraylist.ll)为33KB,809行;而我们的编译结果(result.ll)为27KB,707行。实际上我们的编译器不会处理相同的字符串,而clang是处理了的,如果我们再加上处理相同字符串的功能,编译结果将能进一步缩减尺寸。
    \par 将result.ll在Linux环境下,用\lstinline[language=sh]{clang -o result result.ll}将其编译为二进制文件,然后执行\lstinline[language=sh]{./result},输出结果如下。这与clang直接编译arraylist.c的结果表现一致。可以看到各个功能都被正确地执行了,特别是排序,涉及比较复杂的过程,这些正确的结果有理由让我们相信我们的编译结果是正确的。
    \begin{verbatim}
Start test ArrayList.

> Create an ArrayList.
>> Action: <empty>
>>> Returned: 0x9184008

> Initialize the ArrayList.
>> Action: list1.init()
>>> Returned: 0x0

> Create an ArrayList.
>> Action: <empty>
>>> Returned: 0x9184030

> Initialize the ArrayList.
>> Action: list2.init()
>>> Returned: 0x0

> Insert to the ArrayList.
>> Action: list1[0]=5
>>> Returned: 0x0

> Insert to the ArrayList.
>> Action: list1[1]=2
>>> Returned: 0x0

> Insert to the ArrayList.
>> Action: list1[2]=8
>>> Returned: 0x0

> Insert to the ArrayList.
>> Action: list1[3]=9
>>> Returned: 0x0

> Remove from the ArrayList by value
>> Action: list1.removeValue(9)
>>> Returned: 0x0

> Insert to the ArrayList.
>> Action: list1[3]=0
>>> Returned: 0x0

> Insert to the ArrayList.
>> Action: list1[4]=6
>>> Returned: 0x0

> Insert to the ArrayList.
>> Action: list2[0]=7
>>> Returned: 0x0

> Insert to the ArrayList.
>> Action: list2[0]=0
>>> Returned: 0x0

> Insert to the ArrayList.
>> Action: list2[0]=9
>>> Returned: 0x0

> Insert to the ArrayList.
>> Action: list2[0]=2
>>> Returned: 0x0

> Insert to the ArrayList.
>> Action: list2[0]=4
>>> Returned: 0x0

> Sort the combined ArrayList.
>> Action: list1.sort()
>>> Returned: 0x9184008

Sorted list1: 0 2 5 6 8

> Delete the ArrayList.
>> Action: list1
>>> Returned: 0x0

> Delete the ArrayList.
>> Action: list2
>>> Returned: 0x0
    \end{verbatim}

\clearpage
\end{CJK*}
\end{document}


