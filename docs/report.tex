\documentclass[a4paper]{article}
%\documentclass[twocolumn]{article}

\usepackage{graphicx}
\usepackage{listings}
\usepackage{xcolor}
%\usepackage{enumitem}
\usepackage{enumerate}
\usepackage{CJKutf8} %注意这里用的是CJKutf8而不是CJK
\usepackage{indentfirst}

\usepackage{tikz} % 画流程图用的
\usepackage{qtree}
\usepackage{indentfirst}%英文首行缩进
\usepackage{fancyhdr} % 排版格式
\usepackage{hyphenat} % 单词断字
\usepackage{amsmath} % for {aligned}, 公式换行
\usepackage{multicol}% 多栏排版
\usepackage{balance}% 双栏最后一页对齐
\usepackage{subfigure}% 多图
\usepackage{booktabs}% 表格画线,\toprule, \midrule, \bottomrule
\usepackage{ulem}
%\usepackage{clrscode}

% in texlive-science
\usepackage{algorithm}
\usepackage{algpseudocode}% an improvement from algorithmicx for algorithmic

\usepackage{setspace}
 
\usepackage[utf8]{inputenc}


\usepackage{xspace}

%======= XXX 要编译两遍才能有标签和引用等效果 =====%
 
\usepackage[top=2.54cm,bottom=2.54cm,left=3.17cm,right=3.17cm]{geometry} % a4paper standard
\usepackage[unicode=true]{hyperref} %注意这里不能加CJKbookmarks=true,否则会乱码
\usetikzlibrary{arrows,decorations.pathmorphing,backgrounds,positioning,fit,automata,trees}

\hypersetup{
    pdfauthor={ouoline},
    %pdftitle={test},
    %pdfsubject={Subject},
    %pdfkeywords={Keyword1, Keyword2, ...},
    %pdfcreator={LaTeX with hyperref package},
    %pdfproducer = {dvips + ps2pdf},
    %bookmarksnumbered=true,
    %colorlinks=no,
    pdfborder={0 0 0},
    %bookmarksopen=true,
}
%------------------------------------------------------------------%
 
\setcounter{secnumdepth}{5} % 编号的深度,4 表示到 paragraph 一级
%\setcounter{tocdepth}{4} % 目录中的深度
 
%------------------------------------------------------------------%
\usepackage{color}
\definecolor{lightgray}{rgb}{.9,.9,.9}
\definecolor{darkgray}{rgb}{.4,.4,.4}
\definecolor{purple}{rgb}{0.65, 0.12, 0.82}

\lstdefinelanguage{JavaScript}{
  keywords={typeof, new, true, false, catch, function, return, null, catch, switch, var, if, in, while, do, else, case, break},
  keywordstyle=\color{blue}\bfseries,
  ndkeywords={class, export, boolean, throw, implements, import, this},
  ndkeywordstyle=\color{darkgray}\bfseries,
  identifierstyle=\color{black},
  sensitive=false,
  comment=[l]{//},
  morecomment=[s]{/*}{*/},
  commentstyle=\color{purple}\ttfamily,
  stringstyle=\color{red}\ttfamily,
  morestring=[b]',
  morestring=[b]"
}

\lstset{% general command to set parameter(s)
        language={JavaScript},
        %numbers=left,
        basicstyle=\tt, % 默认对所有字符使用等宽字体
        keywordstyle=\color{blue},%\bfseries\underbar% underlined bold black keywords
        identifierstyle=,           % nothing happens
        stringstyle=\color{purple},
        commentstyle=\color{gray},
        escapechar=`,
        showstringspaces=false,     % no special string spaces
        breaklines=true, % 自动换行
    }


%------------------------------------------------------------------%
 
% 自己定义新命令,参数依次是
% \newcommand{新命令名称(带反斜线)}[参数个数(最多9个)]{命令定义}
% 实际上相当于宏替换
% \newcommand{\sayhelloto}[1]{hello,#1}
\newcommand{\template}[3] {
    \item \textbf{#1}
    \begin{enumerate}
        \item[\textbf{#2}] #3 
    \end{enumerate}
}
\newcommand{\jie}[2]{\template{#1}{解}{#2}}
\newcommand{\zheng}[2]{\template{#1}{证明}{#2}}
 
%------------------------------------------------------------------%
 
% 放在导言区,设置全局行距
\linespread{1.6}
 
% 放在导言区,公式编号和章节相关
%\makeatletter % `@' now normal ``letter''
%\@addtoreset{equation}{section}
%\makeatother % `@' is restored as ``non-letter''
%\renewcommand\theequation{\oldstylenums{\thesection}%
%.\oldstylenums{\arabic{equation}}}


\begin{document}

\begin{CJK*}{UTF8}{gbsn}
    \CJKindent
    \setlength{\parindent}{2em} % no indent
 
    \pagestyle{fancy}
  
    %\begin{center}
    %\Huge{title}
    %\vspace{25pt} % 25pt between title and text
    %\end{center}
    \title{\huge{计算机与网络体系结构(2)}\\\Large{编译原理}\\{\large 大作业:简单编译器的实现}}
    \author{
    软件11\hspace{10pt}陈 璐\hspace{10pt}2011013249\\
    软件12\hspace{10pt}王肖佑\hspace{10pt}2011013273\\
    软件11\hspace{10pt}陈华榕\hspace{10pt}2011013236
    }
    \date{\today}
    \maketitle
    \tableofcontents    
    \newpage

    \section{实验背景}
    \subsection{实验环境}
    如需复现实验结果请准备一台PC机并安装Windows/OS X/Linux等桌面系统,然后安装$>=0.10.24$版本的\textit{Node}。

    \subsection{目录结构}
{
\tikzstyle{every node}=[draw=black,thick,anchor=west]
\tikzstyle{selected}=[draw=red,fill=red!30]
\tikzstyle{optional}=[dashed,fill=gray!50]
\begin{tikzpicture}[%
  grow via three points={one child at (0.5,-0.7) and
  two children at (0.5,-0.7) and (0.5,-1.4)},
  edge from parent path={(\tikzparentnode.south) |- (\tikzchildnode.west)}]
  \node {2011013236}
    child { node [optional] {source $|$ 源码}
        child { node [optional] {BibJS}
            child { node [optional] {lib $|$ 库文件}
                child { node [optional] {automata $|$ 自动机\footnotemark[1]}
                    child { node {inner-data.js $|$ 定义自动机的数据}}
                    child { node {runner.js $|$ 自动机的运行器}}
                }
                child [missing] {}
                child [missing] {}
                child { node [optional] {scanner $|$ 词法单元扫描器}
                    child { node {token-types.js $|$ 支持的词法单元类型}}
                    child { node {char-token.js $|$ 字符单元}}
                    child { node {token.js $|$ 词法单元}}
                    child { node {scanner.js $|$ 扫描器}}
                }
                child [missing] {}
                child [missing] {}
                child [missing] {}
                child [missing] {}
                child { node {bibdad.js $|$ Bib解析器}}
            }
                child [missing] {}
                child [missing] {}
                child [missing] {}
                child [missing] {}
                child [missing] {}
                child [missing] {}
                child [missing] {}
                child [missing] {}
                child [missing] {}
            child { node [optional] {tests $|$ 自动化测试}
                child { node [optional] {data $|$ 测试数据}
                    child { node {example.bib $|$ 测试文件}}
                }
                child [missing] {}
                child { node [selected] {test-bib.js $|$ 测试执行入口,可通过npm调用}}
            }
                child [missing] {}
                child [missing] {}
                child [missing] {}
            child { node {index.js $|$ 库入口}}
    child { node {... $|$ 其他文件,包括定义库的package.json等}}
        }
    }
                child [missing] {}
                child [missing] {}
                child [missing] {}
                child [missing] {}
                child [missing] {}
                child [missing] {}
                child [missing] {}
                child [missing] {}
                child [missing] {}
                child [missing] {}
                child [missing] {}
                child [missing] {}
                child [missing] {}
                child [missing] {}
                child [missing] {}
                child [missing] {}
                child [missing] {}
    child { node {report.pdf $|$ 本报告}}
    child { node {README.txt $|$ 请先阅读此文件}}
    child { node {result.txt $|$ 本实验所给example.bib的解析结果}}
;
\end{tikzpicture}
}
\footnotetext[1]{无$\epsilon$转移的单状态非确定型有限自动机}

    \subsection{完成情况与测试}
    \label{sec:abouttest}
    本实验中,通过Node.js完成了一个Bib文件解析库BibJS,可作为包供第三方Node.js应用引入。还借助这个库完成了符合实验要求的测试程序。由于所用编程语言是脚本语言,因此不需要提交可执行的版本。
    \par 本次作业提交在了Github上,在作业提交的截止时间前以私有库存在,作业提交截止时间后就修改为公开库,git版本库的地址为:\href{https://github.com/Epsirom/BibJS}{https://github.com/Epsirom/BibJS}。
    \par 如果在安装了npm的机器上进行测试,可以直接在BibJS目录下运行\lstinline[language=sh]{npm test},将以./tests/data/example.bib为输入,处理结果将输出到./tests/data/result.txt。
    \par 如果在未安装npm的机器上进行测试,请以BibJS目录为工作目录运行./tests/test-bib.js,即在BibJS目录下运行\lstinline{node ./tests/test-bib.js}。
    \par 运行测试程序,会将带调试信息的结果输出在命令行,而输出到result.txt的则不带调试信息只保留有效的解析结果。

    \section{实验分析}
    \subsection{整体思路}
    我将本次实验目标分为两部分:第一部分通过构建自动机A由字符串分析词法单元;第二部分通过构建自动机B由词法单元构建Bib数据对象。进行Bib解析时,先通过自动机A分析出一个词法单元,将该词法单元作为自动机B的一次输入,然后再通过自动机A分析出一个词法单元作为自动机B的新一次输入,不断进行下去直到发生错误或完整解析了一个Bib对象。
    \par 两部分所用的自动机都是“无$\epsilon$转移的单状态非确定型有限自动机”,这种自动机是我自行定义的,兼顾DFA的实现简单和NFA的一部分灵活性,能比较方便地应用在本次实验中。
    \par 实际上本实验的要求可以通过先编写一个涵盖自动机和词法分析的Bib解析库,然后借助该库对输入文件进行分析并按格式输出结果,这样一种方式来完成。我正是这么做的。

    \subsection{词法、语法分析}

    \subsubsection{Antlr工具}

    \subsubsection{抽象语法分析树AST}

    \subsection{语法及语义检查}
    \subsubsection{实现概况}

    \subsubsection{变量作用域检查}

    \subsection{目标代码生成}
    \subsubsection{实现概况}

    \subsubsection{自顶向下基于模板的目标代码生成}

    \subsubsection{目前支持的LLVM语言特性}

    \subsection{自动化测试}
    \subsubsection{实现概况}
    完成了一个自动化测试实例BibJS/tests/test-bib.js,运行方法见\ref{sec:abouttest}。该测试中会依照实验要求将example.bib读入,然后将解析结果输出至result.txt。这两个文件都放在BibJS/tests/data目录下。

    \subsubsection{自动化测试步骤}
    在该自动化测试中,先从文件中读取输入字符串,然后用它构建BibDad对象,接着不断调用该对象analyze方法,若该方法返回true,说明解析成功,就进行validate,如果仍通过validate就调用output将结果输出到文件。
    \par 若analyze返回false,说明解析失败,判断此时是完成了解析还是解析过程中出错,如果是解析过程中出错,就打印错误信息(会打印出在输入串的第几个字节解析出错,并打印该字节附近以15字节为半径的字符串,空白符会全部替换为空格),错误信息的显示效果还算不错。

    \section{实验结果}
    本实验提供的example.bib的解析结果见result.txt。
    \par 在控制台的输出与result.txt不同,还带有调试信息,您可以试试按\ref{sec:abouttest}的步骤运行。
    \par 也可以试试修改BibJS/tests/data/example.bib使之不符合Bib规范,再运行看看控制台是怎样输出错误信息的。

\clearpage
\end{CJK*}
\end{document}


